\subsection{Análise de requisitos}
\subsubsection{Requisitos funcionais}
Os requisitos funcionais do sistema desenvolvido serão: 

\begin{itemize}
	\item O cadastro de administradores, clientes, produtos, serviços e agendamentos;
	\item A alteração de administradores, clientes, produtos, serviços e agendamentos;
	\item A listagem de administradores, clientes, produtos, serviços e agendamentos;
	\item A listagem de produtos e serviços para acesso público;
	\item Autenticação;
	\item Cálculos monetários;
	\item Efetuação de pagamentos de serviços agendados e concluídos;
\end{itemize}

\subsubsection{Requisitos não funcionais}
Os requisitos não funcionais do sistema serão:

\begin{itemize}
	\item desenvolvimento no padrão MVC;
	\item persistência a dados;
	\item o front-end será no modelo de aplicação de página única;
	\item o front-end será desenvolvido no modelo mobile first;
	\item o projeto será homologado nas resoluções de 576px, 768px e 1200px;
	\item implantações do servidor e do front-end em serviços diferentes;
\end{itemize}

\subsection{Tecnologias adotadas}
\subsubsection{Modelagem}
\begin{itemize}
	\item Astah foi a ferramenta utilizada para o desenvolvimento dos diagramas UML.
\end{itemize}
\subsubsection{Backend}
\begin{itemize}
	\item Springboot ferramenta para o desenvolvimento do servidor.
	\item Java linguagem de programação utilizada no servidor.
	\item PostgreSQL sistema de gerenciamento de banco de dados.
	\item Postman software em que as requisições do sistema serão testadas.
\end{itemize}
\subsubsection{Frontend}
\begin{itemize}
	\item React framework javascript que ficará responsável pela parte visual do sistema.
	\item SASS tecnologia para o desenvolvimento de estilos personalizados.
	\item Bootstrap framework CSS.
\end{itemize}
\subsubsection{Versionamento}
\begin{itemize}
	\item Git
	\item GitHub
\end{itemize}
\subsubsection{Domínio e hospedagem}
\begin{itemize}
	\item Heroku encarregado em hospedar o servidor.
	\item Netlify ou vercel responsáveis em hospedar a parte visual.
	\item Domínio ficará a critério do administrador do sistema.
\end{itemize}

