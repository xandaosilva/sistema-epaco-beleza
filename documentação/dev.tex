\subsection{Análise de requisitos}
\subsubsection{Requisitos funcionais}
Os requisitos funcionais do sistema desenvolvido serão o gerenciamento de administradores, clientes, produtos, serviços e agendamentos.
Todos os cálculos deverão ser realizados de forma automatizada.
As informações poderão ser detalhadas e selecionadas de acordo com os administradores e deverão ser exibidas de acordo com o perfil de usuário, por exemplo, informações de cadastro de cada cliente poderão ser acessadas apenas pelos administradores do sistema.
\subsubsection{Requisitos não funcionais}
Os requisitos não funcionais do sistema serão o seu desenvolvimento no padrão de projeto MVC, persistência a dados com sgbd e a parte visual será desenvolvida no modo de aplicação de página única e com responsividade ( desenvolvimento mobile first ).
\subsection{Tecnologias adotadas}
\subsubsection{Modelagem}
\begin{itemize}
	\item Astah foi a ferramenta utilizada para o desenvolvimento dos diagramas UML.
\end{itemize}
\subsubsection{Backend}
\begin{itemize}
	\item Springboot ferramenta adotada para o desenvolvimento do backend.
	\item Java será a linguagem de programação utilizada no servidor e nos controladores.
	\item Postgres sistema de gerenciamento de banco de dados.
	\item Postman software terá por finalidade testar as requisições http do sistema.
\end{itemize}
\subsubsection{Frontend}
\begin{itemize}
	\item React frmaework javascript que ficará responsável pela parte visual e do gerenciamento de rotas do sistema.
	\item SASS tecnologia para o desenvolvimento de estilos personalizados.
	\item Bootstrap framework CSS.
\end{itemize}
\subsubsection{Versionamento}
\begin{itemize}
	\item Git
	\item GitHub
\end{itemize}
\subsubsection{Domínio e hospedagem}
\begin{itemize}
	\item Heroku
	\item Netlify
	\item Domínio ficará a critério do administrador do sistema.
\end{itemize}

